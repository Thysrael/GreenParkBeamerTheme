\documentclass{ctexbeamer}

\usetheme{gpark}
% meta-data
\title{BUAA Beamer 主题 \\ Green Park 绿园}
\subtitle{使用 \LaTeX\ 制作幻灯片}
\author{\href{https://github.com/Thysrael}{Thysrael}}
\date{创建于 2024 年 6 月 11 日}
\titlebackground{images/titlebackground}
\background{images/background}

\begin{document}

\maketitle

% 声明页
\begin{frame}{}
  “绿园 GreenPark”是由北航六系同学开发的一款非官方的 \LaTeX Beamer 主题,它具有简约现代的设计特点。
  
  GreenPark 是基于 \chref{https://www.overleaf.com/latex/templates/sintef-presentation/jhbhdffczpnx}{SINTEF Presentation} 和 \chref{https://github.com/liu-qilong/Beamer-LaTeX-Themes}{Beamer-LaTeX-Themes} 的二次开发,GreenPark 去掉了 SINTEF 的一些 deprecated 的特征,并增加了中文支持。感谢这些开发者高品味和优雅的设计。封面素材来自北航姜海洋同学的摄影作品。背景图片来自北航新闻中心,均已取得相关授权。感谢他们的慷慨。

  本主题采用 \chref{https://creativecommons.org/licenses/by/4.0/deed.zh-hans}{CC-BY-4.0},您可以自由地共享和演绎,只需要给出适当的署名。

  本主题目前已经结束开发\textsl{(其实只是为了在离开北航前留个印迹)}。但是后续或许会继续开发。欢迎提交 issue/pr 。

\end{frame}

\section{Introduction}

% % \textbf{彩色粗体 ctextbf} \textbf{普通粗体 textbf},\ctextsl{彩色斜体 ctextsl} \textsl{普通斜体 textsl} ,\cemph{彩色打印文本 emph} \emph{普通打印文本 emph}
% \textbf{普通粗体 textbf} \ctextbf{彩色粗体 textbf},\textsl{彩色斜体 textsl} ,\emph{彩色打印文本 emph}
% 
% \begin{minted}{python}
%   def main():
%   print("Hello, world!");
%   return 0
% \end{minted}

\begin{frame}{Beamer for SINTEF slides}{\thesection \, \secname}

  \begin{itemize}
  \item We assume you can use \LaTeX; if you cannot,
    \chref{http://en.wikibooks.org/wiki/LaTeX/}{you can learn it here}
  \item Beamer is one of the most popular and powerful document
    classes for presentations in \LaTeX
  \item Beamer has also a detailed
    \chref{http://www.ctan.org/tex-archive/macros/latex/contrib/beamer/doc/beameruserguide.pdf}{user manual}
  \item Here we will present only the most basic features to get you up to speed
  \end{itemize}

\end{frame}

\begin{frame}{Beamer vs. PowerPoint}
  Compared to PowerPoint, using \LaTeX\ is better because:
  \begin{itemize}
  \item It is not What-You-See-Is-What-You-Get, but
    What-You-\emph{Mean}-Is-What-You-Get:\\
    you write the content, the computer does the typesetting
  \item Produces a \texttt{pdf}: no problems with fonts, formulas,
    program versions
  \item Easier to keep consistent style, fonts, highlighting, etc.
  \item Math typesetting in \TeX\ is the best:
    \begin{equation*}
      \mathrm{i}\,\hslash\frac{\partial}{\partial t} \Psi(\mathbf{r},t) =
      -\frac{\hslash^2}{2\,m}\nabla^2\Psi(\mathbf{r},t)
      + V(\mathbf{r})\Psi(\mathbf{r},t)
    \end{equation*}

  \end{itemize}
\end{frame}

\section{Editing}

\begin{frame}[fragile]{Selecting the Class}
  After the last update to the graphic profile, the \texttt{sintef} theme for
  Beamer has been updated into a full-fledged class.
  To start working with \texttt{sintefbeamer}, start a \LaTeX\ document with the
  preamble:
  \begin{block}{Minimum SINTEF Beamer Document}
    \begin{lstlisting}[language=TeX]
      \documentclass{sintefbeamer}
      \begin{document}
      \begin{frame}{Hello, world!}
      \end {frame}
    \end{document}
  \end{lstlisting}
\end{block}
\end{frame}

\begin{frame}[fragile]{Title page}
  To set a typical title page, you call some commands in the preamble:
  \begin{block}{The Commands for the Title Page}
    \begin{lstlisting}[language=TeX]
      \title{Sample Title}
      \subtitle{Sample subtitle}
      \author{First Author, Second Author}
      \date{Defaults to today's}
    \end{lstlisting}
  \end{block}
  You can then write out the title page with \verb|\maketitle|.

  You can set a different background image than the default one with the
  \verb|\titlebackground| command, set before \verb|\maketitle|.

  In the \texttt{backgrounds} folder, you can find a lot of standard backgrounds
  for SINTEF presentation title pages.

\end{frame}

\begin{frame}[fragile]{Writing a Simple Slide}
  \framesubtitle{It's really easy!}
  \begin{itemize}[<+->]
  \item A typical slide has bulleted lists
  \item These can be uncovered in sequence
  \end{itemize}
  \begin{block}{Code for a Page with an Itemised List}<+->
    \begin{lstlisting}[language=TeX]
      \begin{frame}
        \frametitle{Writing a Simple Slide}
        \framesubtitle{It's really easy!}
        \begin{itemize}[<+->]
        \item A typical slide has bulleted lists
        \item These can be uncovered in sequence
        \end{itemize}
      \end{frame}\end{lstlisting}
  \end{block}
\end{frame}

\begin{frame}[fragile]{Adding images}
  \begin{columns}
    \begin{column}{0.7\textwidth}
      Adding images works like in normal \LaTeX:
      \begin{block}{Code for Adding Images}
        \begin{lstlisting}[language=TeX]
          \usepackage{graphicx}
          % ...
          \includegraphics
          [width=\textwidth]{images/default}
        \end{lstlisting}
      \end{block}
    \end{column}
    \begin{column}{0.3\textwidth}
      \includegraphics
      [width=\textwidth]{images/default}\\
    \end{column}
  \end{columns}
\end{frame}

\begin{frame}[fragile]{Splitting in Columns}
  Splitting the page is easy and common;
  typically, one side has a picture and the other text:
  \begin{columns}
    \begin{column}{0.6\textwidth}
      This is the first column
    \end{column}
    \begin{column}{0.3\textwidth}
      And this the second
    \end{column}
  \end{columns}
  \begin{block}{Column Code}
    \begin{lstlisting}[language=TeX]
      \begin{columns}
        \begin{column}{0.6\textwidth}
          This is the first column
        \end{column}
        \begin{column}{0.3\textwidth}
          And this the second
        \end{column}
        % There could be more!
      \end{columns}
    \end{lstlisting}
  \end{block}
\end{frame}

\begin{frame}[fragile]
  \frametitle{Fonts}
  \begin{itemize}
  \item The paramount task of fonts is being readable
  \item There are good ones...
    \begin{itemize}
    \item {\textrm{Use serif fonts only with high-definition projectors}}
    \item {\textsf{Use sans-serif fonts otherwise (or if you simply prefer them)}}
    \end{itemize}
  \item ... and not so good ones:
    \begin{itemize}
    \item {\texttt{Never use monospace for normal text}}
    \item {\frakfamily Gothic, calligraphic or weird fonts: should always: be
        avoided}
    \end{itemize}
  \end{itemize}
\end{frame}

\begin{frame}[fragile]{Look}
  \begin{itemize}
  \item To change the colour of the title dash, give one of the class options
    \texttt{cyandash} (default), \texttt{greendash}, \texttt{magentadash},
    \texttt{yellowdash}, or \texttt{nodash}.
  \item To change between the light and dark themes, give the class options
    \texttt{light} (default) or \texttt{dark}. It is not possible to switch
    theme for one slide because of the design of Beamer---and it's probably a
    good thing.
  \item To insert a final slide, use \verb|\backmatter|.
  \item The aspect ratio defaults to 16:9, but you can change it to 4:3 for old
    projectors by passing the class option \texttt{aspectratio=43}; any other
    values accepted by Beamer are also possible.
  \end{itemize}
\end{frame}

\section{Summary}

\begin{frame}
  \frametitle{Good Luck!}
  \begin{itemize}
  \item Enough for an introduction! You should know enough by now
  \item If you have corrections or suggestions,
    \chref{mailto:federico.zenith@sintef.no}{send them to me!}
  \end{itemize}
\end{frame}

\backmatter 

\end{document}
